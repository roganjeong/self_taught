% Options for packages loaded elsewhere
\PassOptionsToPackage{unicode}{hyperref}
\PassOptionsToPackage{hyphens}{url}
%
\documentclass[
]{article}
\title{SM\_Example1.R}
\author{rogans}
\date{2021-10-15}

\usepackage{amsmath,amssymb}
\usepackage{lmodern}
\usepackage{iftex}
\ifPDFTeX
  \usepackage[T1]{fontenc}
  \usepackage[utf8]{inputenc}
  \usepackage{textcomp} % provide euro and other symbols
\else % if luatex or xetex
  \usepackage{unicode-math}
  \defaultfontfeatures{Scale=MatchLowercase}
  \defaultfontfeatures[\rmfamily]{Ligatures=TeX,Scale=1}
\fi
% Use upquote if available, for straight quotes in verbatim environments
\IfFileExists{upquote.sty}{\usepackage{upquote}}{}
\IfFileExists{microtype.sty}{% use microtype if available
  \usepackage[]{microtype}
  \UseMicrotypeSet[protrusion]{basicmath} % disable protrusion for tt fonts
}{}
\makeatletter
\@ifundefined{KOMAClassName}{% if non-KOMA class
  \IfFileExists{parskip.sty}{%
    \usepackage{parskip}
  }{% else
    \setlength{\parindent}{0pt}
    \setlength{\parskip}{6pt plus 2pt minus 1pt}}
}{% if KOMA class
  \KOMAoptions{parskip=half}}
\makeatother
\usepackage{xcolor}
\IfFileExists{xurl.sty}{\usepackage{xurl}}{} % add URL line breaks if available
\IfFileExists{bookmark.sty}{\usepackage{bookmark}}{\usepackage{hyperref}}
\hypersetup{
  pdftitle={SM\_Example1.R},
  pdfauthor={rogans},
  hidelinks,
  pdfcreator={LaTeX via pandoc}}
\urlstyle{same} % disable monospaced font for URLs
\usepackage[margin=1in]{geometry}
\usepackage{color}
\usepackage{fancyvrb}
\newcommand{\VerbBar}{|}
\newcommand{\VERB}{\Verb[commandchars=\\\{\}]}
\DefineVerbatimEnvironment{Highlighting}{Verbatim}{commandchars=\\\{\}}
% Add ',fontsize=\small' for more characters per line
\usepackage{framed}
\definecolor{shadecolor}{RGB}{248,248,248}
\newenvironment{Shaded}{\begin{snugshade}}{\end{snugshade}}
\newcommand{\AlertTok}[1]{\textcolor[rgb]{0.94,0.16,0.16}{#1}}
\newcommand{\AnnotationTok}[1]{\textcolor[rgb]{0.56,0.35,0.01}{\textbf{\textit{#1}}}}
\newcommand{\AttributeTok}[1]{\textcolor[rgb]{0.77,0.63,0.00}{#1}}
\newcommand{\BaseNTok}[1]{\textcolor[rgb]{0.00,0.00,0.81}{#1}}
\newcommand{\BuiltInTok}[1]{#1}
\newcommand{\CharTok}[1]{\textcolor[rgb]{0.31,0.60,0.02}{#1}}
\newcommand{\CommentTok}[1]{\textcolor[rgb]{0.56,0.35,0.01}{\textit{#1}}}
\newcommand{\CommentVarTok}[1]{\textcolor[rgb]{0.56,0.35,0.01}{\textbf{\textit{#1}}}}
\newcommand{\ConstantTok}[1]{\textcolor[rgb]{0.00,0.00,0.00}{#1}}
\newcommand{\ControlFlowTok}[1]{\textcolor[rgb]{0.13,0.29,0.53}{\textbf{#1}}}
\newcommand{\DataTypeTok}[1]{\textcolor[rgb]{0.13,0.29,0.53}{#1}}
\newcommand{\DecValTok}[1]{\textcolor[rgb]{0.00,0.00,0.81}{#1}}
\newcommand{\DocumentationTok}[1]{\textcolor[rgb]{0.56,0.35,0.01}{\textbf{\textit{#1}}}}
\newcommand{\ErrorTok}[1]{\textcolor[rgb]{0.64,0.00,0.00}{\textbf{#1}}}
\newcommand{\ExtensionTok}[1]{#1}
\newcommand{\FloatTok}[1]{\textcolor[rgb]{0.00,0.00,0.81}{#1}}
\newcommand{\FunctionTok}[1]{\textcolor[rgb]{0.00,0.00,0.00}{#1}}
\newcommand{\ImportTok}[1]{#1}
\newcommand{\InformationTok}[1]{\textcolor[rgb]{0.56,0.35,0.01}{\textbf{\textit{#1}}}}
\newcommand{\KeywordTok}[1]{\textcolor[rgb]{0.13,0.29,0.53}{\textbf{#1}}}
\newcommand{\NormalTok}[1]{#1}
\newcommand{\OperatorTok}[1]{\textcolor[rgb]{0.81,0.36,0.00}{\textbf{#1}}}
\newcommand{\OtherTok}[1]{\textcolor[rgb]{0.56,0.35,0.01}{#1}}
\newcommand{\PreprocessorTok}[1]{\textcolor[rgb]{0.56,0.35,0.01}{\textit{#1}}}
\newcommand{\RegionMarkerTok}[1]{#1}
\newcommand{\SpecialCharTok}[1]{\textcolor[rgb]{0.00,0.00,0.00}{#1}}
\newcommand{\SpecialStringTok}[1]{\textcolor[rgb]{0.31,0.60,0.02}{#1}}
\newcommand{\StringTok}[1]{\textcolor[rgb]{0.31,0.60,0.02}{#1}}
\newcommand{\VariableTok}[1]{\textcolor[rgb]{0.00,0.00,0.00}{#1}}
\newcommand{\VerbatimStringTok}[1]{\textcolor[rgb]{0.31,0.60,0.02}{#1}}
\newcommand{\WarningTok}[1]{\textcolor[rgb]{0.56,0.35,0.01}{\textbf{\textit{#1}}}}
\usepackage{graphicx}
\makeatletter
\def\maxwidth{\ifdim\Gin@nat@width>\linewidth\linewidth\else\Gin@nat@width\fi}
\def\maxheight{\ifdim\Gin@nat@height>\textheight\textheight\else\Gin@nat@height\fi}
\makeatother
% Scale images if necessary, so that they will not overflow the page
% margins by default, and it is still possible to overwrite the defaults
% using explicit options in \includegraphics[width, height, ...]{}
\setkeys{Gin}{width=\maxwidth,height=\maxheight,keepaspectratio}
% Set default figure placement to htbp
\makeatletter
\def\fps@figure{htbp}
\makeatother
\setlength{\emergencystretch}{3em} % prevent overfull lines
\providecommand{\tightlist}{%
  \setlength{\itemsep}{0pt}\setlength{\parskip}{0pt}}
\setcounter{secnumdepth}{-\maxdimen} % remove section numbering
\ifLuaTeX
  \usepackage{selnolig}  % disable illegal ligatures
\fi

\begin{document}
\maketitle

\begin{Shaded}
\begin{Highlighting}[]
\DocumentationTok{\#\#\#\#\#\#\#\#\#\#\#\#\#\#\#\#\#\#\#\#\#\#\#\#\#\#\#\#\#\#\#\#\#\#\#\#\#\#\#\#\#\#\#\#}
\CommentTok{\# Statistical Modelling \& Machine Learning \#}
\CommentTok{\#               R Example1                 \#}
\DocumentationTok{\#\#\#\#\#\#\#\#\#\#\#\#\#\#\#\#\#\#\#\#\#\#\#\#\#\#\#\#\#\#\#\#\#\#\#\#\#\#\#\#\#\#\#\#}

\DocumentationTok{\#\#\#\#\#\#\#\#\#\#\#\#\#\#\#\#\#\#\#\#\#\#\#\#\#\#\#\#\#\#\#\#\#\#\#\#\#\#\#\#\#\#\#\#\#\#\#\#\#\#\#\#\#\#\#\#\#\#\#\#\#\#\#\#\#\#\#}
\CommentTok{\# Modelling Example 1:  Nonlinear model with nonconstant variance \#}
\DocumentationTok{\#\#\#\#\#\#\#\#\#\#\#\#\#\#\#\#\#\#\#\#\#\#\#\#\#\#\#\#\#\#\#\#\#\#\#\#\#\#\#\#\#\#\#\#\#\#\#\#\#\#\#\#\#\#\#\#\#\#\#\#\#\#\#\#\#\#\#}

\FunctionTok{library}\NormalTok{(datasets)}
\NormalTok{?attenu}
\FunctionTok{dim}\NormalTok{(attenu)}
\end{Highlighting}
\end{Shaded}

\begin{verbatim}
## [1] 182   5
\end{verbatim}

\begin{Shaded}
\begin{Highlighting}[]
\DocumentationTok{\#\#\#\#\# Linear regression \#\#\#\#\#}
\NormalTok{fit1 }\OtherTok{=} \FunctionTok{lm}\NormalTok{(accel }\SpecialCharTok{\textasciitilde{}}\NormalTok{ mag }\SpecialCharTok{+}\NormalTok{ dist, }\AttributeTok{data=}\NormalTok{attenu)}
\FunctionTok{par}\NormalTok{(}\AttributeTok{mfrow =} \FunctionTok{c}\NormalTok{(}\DecValTok{2}\NormalTok{,}\DecValTok{2}\NormalTok{))}
\FunctionTok{plot}\NormalTok{(fit1)}
\end{Highlighting}
\end{Shaded}

\includegraphics{SM_Example1_files/figure-latex/unnamed-chunk-1-1.pdf}

\begin{Shaded}
\begin{Highlighting}[]
\FunctionTok{summary}\NormalTok{(fit1)}
\end{Highlighting}
\end{Shaded}

\begin{verbatim}
## 
## Call:
## lm(formula = accel ~ mag + dist, data = attenu)
## 
## Residuals:
##      Min       1Q   Median       3Q      Max 
## -0.22045 -0.08175 -0.03330  0.05781  0.55885 
## 
## Coefficients:
##               Estimate Std. Error t value Pr(>|t|)    
## (Intercept) -0.2188439  0.0862918  -2.536   0.0121 *  
## mag          0.0729269  0.0146916   4.964  1.6e-06 ***
## dist        -0.0015488  0.0001705  -9.084  < 2e-16 ***
## ---
## Signif. codes:  0 '***' 0.001 '**' 0.01 '*' 0.05 '.' 0.1 ' ' 1
## 
## Residual standard error: 0.1239 on 179 degrees of freedom
## Multiple R-squared:  0.3163, Adjusted R-squared:  0.3087 
## F-statistic: 41.41 on 2 and 179 DF,  p-value: 1.659e-15
\end{verbatim}

\begin{Shaded}
\begin{Highlighting}[]
\FunctionTok{par}\NormalTok{(}\AttributeTok{mfrow =} \FunctionTok{c}\NormalTok{(}\DecValTok{1}\NormalTok{,}\DecValTok{1}\NormalTok{))}
\CommentTok{\# mag, dist 둘다 유의하다. coefficient of determination 자체는 별로 높지 않다. }
\CommentTok{\# 이 inference는 error term이 normality를 따른다는 가정 하에 세워진 추정이기 때문에 해당 부분에 대한 검토가 필요하다.}

\CommentTok{\#error term의 normality를 확인하기 위해 residual plot을 그려보겠다. {-}\textgreater{}  residual에 패턴이 존재한다. 이분산이 존재하는 것으로 추정.}
\CommentTok{\#분산에 대한 조정을 하기에는 아무런 정보가 없어 mean function부터 추정해야 한다. }
\CommentTok{\#선형이 아닌 회귀모형이 필요한데 아무런 정보가 없는 상태이다. 이렇게 정보가 아무것도 없을 때 사용할 수 있는 것은 GAM이다. }

\DocumentationTok{\#\#\#\#\# GAM \#\#\#\#\#}
\FunctionTok{library}\NormalTok{(gam)}
\end{Highlighting}
\end{Shaded}

\begin{verbatim}
## Loading required package: splines
\end{verbatim}

\begin{verbatim}
## Loading required package: foreach
\end{verbatim}

\begin{verbatim}
## Loaded gam 1.20
\end{verbatim}

\begin{Shaded}
\begin{Highlighting}[]
\NormalTok{fit2 }\OtherTok{=} \FunctionTok{gam}\NormalTok{(accel }\SpecialCharTok{\textasciitilde{}} \FunctionTok{s}\NormalTok{(mag,}\DecValTok{5}\NormalTok{) }\SpecialCharTok{+} \FunctionTok{s}\NormalTok{(dist,}\DecValTok{5}\NormalTok{),}\AttributeTok{data=}\NormalTok{attenu) }\CommentTok{\# Degree of Freedom = 5인 smoothing spline을 각 변수에 적용, 여기서 degrees of freedom은 데이터 얼마나 탄력적으로 반응하나의 정도인 것 같다. }
\NormalTok{fit2 }\OtherTok{=} \FunctionTok{gam}\NormalTok{(accel }\SpecialCharTok{\textasciitilde{}} \FunctionTok{s}\NormalTok{(mag,}\DecValTok{2}\NormalTok{) }\SpecialCharTok{+} \FunctionTok{s}\NormalTok{(dist,}\DecValTok{2}\NormalTok{),}\AttributeTok{data=}\NormalTok{attenu)}
\NormalTok{fit2 }\OtherTok{=} \FunctionTok{gam}\NormalTok{(accel }\SpecialCharTok{\textasciitilde{}} \FunctionTok{s}\NormalTok{(mag,}\DecValTok{3}\NormalTok{) }\SpecialCharTok{+} \FunctionTok{s}\NormalTok{(dist,}\DecValTok{3}\NormalTok{),}\AttributeTok{data=}\NormalTok{attenu)}
\NormalTok{fit2 }\OtherTok{=} \FunctionTok{gam}\NormalTok{(accel }\SpecialCharTok{\textasciitilde{}} \FunctionTok{s}\NormalTok{(mag,}\DecValTok{4}\NormalTok{) }\SpecialCharTok{+} \FunctionTok{s}\NormalTok{(dist,}\DecValTok{4}\NormalTok{),}\AttributeTok{data=}\NormalTok{attenu)}
\FunctionTok{par}\NormalTok{(}\AttributeTok{mfrow=}\FunctionTok{c}\NormalTok{(}\DecValTok{1}\NormalTok{,}\DecValTok{2}\NormalTok{))}
\FunctionTok{plot}\NormalTok{(fit2) }\CommentTok{\# dist는 exponential하게 감소됨이 명확하다. mag는 linear한지 nonlinear한지 명확하지 않기 때문에 test를 해봐야 한다. anova를 해서 F{-}test를 해볼 수 있다. }
\end{Highlighting}
\end{Shaded}

\includegraphics{SM_Example1_files/figure-latex/unnamed-chunk-1-2.pdf}

\begin{Shaded}
\begin{Highlighting}[]
\FunctionTok{par}\NormalTok{(}\AttributeTok{mfrow=}\FunctionTok{c}\NormalTok{(}\DecValTok{1}\NormalTok{,}\DecValTok{1}\NormalTok{))}

\NormalTok{fit2\_1 }\OtherTok{=} \FunctionTok{gam}\NormalTok{(accel }\SpecialCharTok{\textasciitilde{}}\NormalTok{ mag }\SpecialCharTok{+} \FunctionTok{s}\NormalTok{(dist,}\DecValTok{5}\NormalTok{),}\AttributeTok{data=}\NormalTok{attenu) }\CommentTok{\#mag를 비선형으로 가정한 fit2와 선형으로 가정한 fit2\_1 비교}
\FunctionTok{anova}\NormalTok{(fit2,fit2\_1) }\CommentTok{\# p{-}value가 유의하게 나왔으므로 비선형으로 가정하는게 적절하다.}
\end{Highlighting}
\end{Shaded}

\begin{verbatim}
## Analysis of Deviance Table
## 
## Model 1: accel ~ s(mag, 4) + s(dist, 4)
## Model 2: accel ~ mag + s(dist, 5)
##   Resid. Df Resid. Dev      Df  Deviance Pr(>Chi)
## 1       173     1.3916                           
## 2       175     1.4267 -2.0002 -0.035094   0.1129
\end{verbatim}

\begin{Shaded}
\begin{Highlighting}[]
\CommentTok{\# mag: non{-}linear function, dist: exponential function}

\DocumentationTok{\#\#\#\#\#\#\#\#\#\# Nonlinear Model with constant variance \#\#\#\#\#\#\#\#\#\#}
\CommentTok{\# Y = accel, X1 = mag, X2 = dist.}
\CommentTok{\# Nonlinear model: Y = beta1 + beta2*X1 + beta3*X1\^{}2 + beta4*exp({-}beta5*X2).}
\CommentTok{\# GAM을 통해 얻은 정보로 parametric nonlinear regression을 세울 수 있다. 5 parameters}
\CommentTok{\# beta4는 decreasing scale을 잡아주는 역할, beta5는 decreasing rate을 조절하는 역할}
\CommentTok{\# mag에 대해서는 polynomial regression, dist에 대해서는 exponential regression을 취한다. }
\CommentTok{\# dist가 비선형이라 LSE와 같은 잣대로 결정할 수 없다. }

\CommentTok{\#추정한 모델함수 만들기 }
\CommentTok{\# exponential regression 파트 때문에 일반적인 linear regression technique을 통해 beta값들을 추정할 수 없다. }
\NormalTok{f }\OtherTok{=} \ControlFlowTok{function}\NormalTok{(beta,X)}
\NormalTok{\{}
\NormalTok{  X1 }\OtherTok{=}\NormalTok{ X[,}\DecValTok{1}\NormalTok{]; X2 }\OtherTok{=}\NormalTok{ X[,}\DecValTok{2}\NormalTok{]  }
\NormalTok{  beta[}\DecValTok{1}\NormalTok{] }\SpecialCharTok{+}\NormalTok{ beta[}\DecValTok{2}\NormalTok{]}\SpecialCharTok{*}\NormalTok{X1 }\SpecialCharTok{+}\NormalTok{ beta[}\DecValTok{3}\NormalTok{]}\SpecialCharTok{*}\NormalTok{X1}\SpecialCharTok{\^{}}\DecValTok{2} \SpecialCharTok{+}\NormalTok{ beta[}\DecValTok{4}\NormalTok{]}\SpecialCharTok{*}\FunctionTok{exp}\NormalTok{(}\SpecialCharTok{{-}}\NormalTok{beta[}\DecValTok{5}\NormalTok{]}\SpecialCharTok{*}\NormalTok{X2)}
\NormalTok{\}}

\CommentTok{\# Objective function: RSS}
\NormalTok{RSS }\OtherTok{=} \ControlFlowTok{function}\NormalTok{(beta,Y,X) }\FunctionTok{sum}\NormalTok{((Y}\SpecialCharTok{{-}}\FunctionTok{f}\NormalTok{(beta,X))}\SpecialCharTok{\^{}}\DecValTok{2}\NormalTok{)}

\CommentTok{\# Gradient vector of the objective function}
\CommentTok{\# RSS를 각 parameter마다 편미분}
\NormalTok{grv }\OtherTok{=} \ControlFlowTok{function}\NormalTok{(beta,Y,X)}
\NormalTok{\{}
\NormalTok{  X1 }\OtherTok{=}\NormalTok{ X[,}\DecValTok{1}\NormalTok{]; X2 }\OtherTok{=}\NormalTok{ X[,}\DecValTok{2}\NormalTok{]}
\NormalTok{  R }\OtherTok{=}\NormalTok{ Y }\SpecialCharTok{{-}} \FunctionTok{f}\NormalTok{(beta,X)}
  \FunctionTok{c}\NormalTok{(}\SpecialCharTok{{-}}\DecValTok{2}\SpecialCharTok{*}\FunctionTok{sum}\NormalTok{(R), }\SpecialCharTok{{-}}\DecValTok{2}\SpecialCharTok{*}\FunctionTok{sum}\NormalTok{(R}\SpecialCharTok{*}\NormalTok{X1), }\SpecialCharTok{{-}}\DecValTok{2}\SpecialCharTok{*}\FunctionTok{sum}\NormalTok{(R}\SpecialCharTok{*}\NormalTok{X1}\SpecialCharTok{\^{}}\DecValTok{2}\NormalTok{), }\SpecialCharTok{{-}}\DecValTok{2}\SpecialCharTok{*}\FunctionTok{sum}\NormalTok{(R}\SpecialCharTok{*}\FunctionTok{exp}\NormalTok{(}\SpecialCharTok{{-}}\NormalTok{beta[}\DecValTok{5}\NormalTok{]}\SpecialCharTok{*}\NormalTok{X2)), }
    \DecValTok{2}\SpecialCharTok{*}\NormalTok{beta[}\DecValTok{4}\NormalTok{]}\SpecialCharTok{*}\FunctionTok{sum}\NormalTok{(R}\SpecialCharTok{*}\NormalTok{X2}\SpecialCharTok{*}\FunctionTok{exp}\NormalTok{(}\SpecialCharTok{{-}}\NormalTok{beta[}\DecValTok{5}\NormalTok{]}\SpecialCharTok{*}\NormalTok{X2)))  }
\NormalTok{\}}

\CommentTok{\# Optimization}
\NormalTok{X }\OtherTok{=} \FunctionTok{cbind}\NormalTok{(attenu}\SpecialCharTok{$}\NormalTok{mag,attenu}\SpecialCharTok{$}\NormalTok{dist)}
\FunctionTok{colnames}\NormalTok{(X) }\OtherTok{=} \FunctionTok{c}\NormalTok{(}\StringTok{\textquotesingle{}mag\textquotesingle{}}\NormalTok{, }\StringTok{\textquotesingle{}dist\textquotesingle{}}\NormalTok{)}
\NormalTok{Y }\OtherTok{=}\NormalTok{ attenu}\SpecialCharTok{$}\NormalTok{accel}
\NormalTok{ml1 }\OtherTok{=} \FunctionTok{optim}\NormalTok{(}\AttributeTok{par =} \FunctionTok{rep}\NormalTok{(}\FloatTok{0.1}\NormalTok{,}\DecValTok{5}\NormalTok{), }\AttributeTok{fn =}\NormalTok{ RSS, }\AttributeTok{gr=}\NormalTok{grv, }\AttributeTok{method=}\StringTok{\textquotesingle{}BFGS\textquotesingle{}}\NormalTok{, }\AttributeTok{X=}\NormalTok{X, }\AttributeTok{Y=}\NormalTok{Y) }\CommentTok{\#  par = initial Beta\textquotesingle{}s, fn = optimize할 objective function, gr = gradient vector를 지정하는 함수. "BFGS" is a quasi{-}Newton method.}
\CommentTok{\# gradient vector는 objective function을 각 parameter에 대해서 1차 미분을 한 것이다.  따로 지정 안해줘도 알아서 numerical하게 값을 찾아준다. 그러나 따로 gradient vector를 계산해서 넣어주면 더 잘 돌아간다. Hessian Matrix도 넣어주면 좋긴 하다. }
\NormalTok{ml1}
\end{Highlighting}
\end{Shaded}

\begin{verbatim}
## $par
## [1] -1.49610172  0.41779875 -0.02823150  0.45349508  0.04482179
## 
## $value
## [1] 1.332736
## 
## $counts
## function gradient 
##      121       37 
## 
## $convergence
## [1] 0
## 
## $message
## NULL
\end{verbatim}

\begin{Shaded}
\begin{Highlighting}[]
\CommentTok{\# ml1으로 새로운 beta coefficients들을 구하였으므로 이것에 대한 residual check 역시 해야 한다. }

\NormalTok{beta.hat }\OtherTok{=}\NormalTok{ ml1}\SpecialCharTok{$}\NormalTok{par}
\NormalTok{beta.hat}
\end{Highlighting}
\end{Shaded}

\begin{verbatim}
## [1] -1.49610172  0.41779875 -0.02823150  0.45349508  0.04482179
\end{verbatim}

\begin{Shaded}
\begin{Highlighting}[]
\CommentTok{\# Fitted value}
\NormalTok{Yhat }\OtherTok{=} \FunctionTok{f}\NormalTok{(beta.hat,X)}

\CommentTok{\# Residual plot}
\NormalTok{r }\OtherTok{=}\NormalTok{ Y }\SpecialCharTok{{-}}\NormalTok{ Yhat}
\FunctionTok{par}\NormalTok{(}\AttributeTok{mfrow=}\FunctionTok{c}\NormalTok{(}\DecValTok{1}\NormalTok{,}\DecValTok{1}\NormalTok{))}
\FunctionTok{plot}\NormalTok{(Yhat,r,}\AttributeTok{ylim=}\FunctionTok{c}\NormalTok{(}\SpecialCharTok{{-}}\FloatTok{0.5}\NormalTok{,}\FloatTok{0.5}\NormalTok{))}
\FunctionTok{lines}\NormalTok{(}\FunctionTok{c}\NormalTok{(}\SpecialCharTok{{-}}\DecValTok{10}\NormalTok{,}\DecValTok{10}\NormalTok{),}\FunctionTok{c}\NormalTok{(}\DecValTok{0}\NormalTok{,}\DecValTok{0}\NormalTok{),}\AttributeTok{col=}\StringTok{\textquotesingle{}red\textquotesingle{}}\NormalTok{)}
\end{Highlighting}
\end{Shaded}

\includegraphics{SM_Example1_files/figure-latex/unnamed-chunk-1-3.pdf}

\begin{Shaded}
\begin{Highlighting}[]
\CommentTok{\# Mean pattern은 없어지는 것으로 나온다. }
\CommentTok{\# Linearly increasing variance pattern.}
\CommentTok{\#residual의 분산이 점점 커지는 패턴을 가짐을 확인할 수 있음. 등분산 X}
\CommentTok{\# 이럴 경우 Linear Variance Function을 사용하는게 적절하다.}

\DocumentationTok{\#\#\#\#\#\#\#\#\# Nonlinear model with nonconstant variance \#\#\#\#\#\#\#\#\#\#}

\CommentTok{\# To check whether a matrix is singular or not}
\CommentTok{\# install.packages(\textquotesingle{}matrixcalc\textquotesingle{}) }
\FunctionTok{library}\NormalTok{(matrixcalc)}

\CommentTok{\# Objective function for mean function: Genearalized least square method.}
\NormalTok{obj.mean }\OtherTok{=} \ControlFlowTok{function}\NormalTok{(beta,Y,X,S) }\FunctionTok{t}\NormalTok{(Y}\SpecialCharTok{{-}}\FunctionTok{f}\NormalTok{(beta,X)) }\SpecialCharTok{\%*\%} \FunctionTok{solve}\NormalTok{(S) }\SpecialCharTok{\%*\%}\NormalTok{ (Y}\SpecialCharTok{{-}}\FunctionTok{f}\NormalTok{(beta,X)) }\CommentTok{\# solve(S) = the inverse of S}
\CommentTok{\# S: Covariance matrix}
\CommentTok{\#5주차 1차시 52분 19초에 나오는 식을 구현한 것}

\CommentTok{\# Gradient vector of the objective function}
\NormalTok{gr.mean }\OtherTok{=} \ControlFlowTok{function}\NormalTok{(beta,Y,X,S)}
\NormalTok{\{}
\NormalTok{  sigma2 }\OtherTok{=} \FunctionTok{diag}\NormalTok{(S)}
\NormalTok{  X1 }\OtherTok{=}\NormalTok{ X[,}\DecValTok{1}\NormalTok{]; X2 }\OtherTok{=}\NormalTok{ X[,}\DecValTok{2}\NormalTok{]}
\NormalTok{  R }\OtherTok{=}\NormalTok{ Y }\SpecialCharTok{{-}} \FunctionTok{f}\NormalTok{(beta,X)}
  \FunctionTok{c}\NormalTok{(}\SpecialCharTok{{-}}\DecValTok{2}\SpecialCharTok{*}\FunctionTok{sum}\NormalTok{(R}\SpecialCharTok{/}\NormalTok{sigma2), }\SpecialCharTok{{-}}\DecValTok{2}\SpecialCharTok{*}\FunctionTok{sum}\NormalTok{(R}\SpecialCharTok{*}\NormalTok{X1}\SpecialCharTok{/}\NormalTok{sigma2), }\SpecialCharTok{{-}}\DecValTok{2}\SpecialCharTok{*}\FunctionTok{sum}\NormalTok{(R}\SpecialCharTok{*}\NormalTok{X1}\SpecialCharTok{\^{}}\DecValTok{2}\SpecialCharTok{/}\NormalTok{sigma2), }
    \SpecialCharTok{{-}}\DecValTok{2}\SpecialCharTok{*}\FunctionTok{sum}\NormalTok{(R}\SpecialCharTok{*}\FunctionTok{exp}\NormalTok{(}\SpecialCharTok{{-}}\NormalTok{beta[}\DecValTok{5}\NormalTok{]}\SpecialCharTok{*}\NormalTok{X2)}\SpecialCharTok{/}\NormalTok{sigma2), }
    \DecValTok{2}\SpecialCharTok{*}\NormalTok{beta[}\DecValTok{4}\NormalTok{]}\SpecialCharTok{*}\FunctionTok{sum}\NormalTok{(R}\SpecialCharTok{*}\NormalTok{X2}\SpecialCharTok{*}\FunctionTok{exp}\NormalTok{(}\SpecialCharTok{{-}}\NormalTok{beta[}\DecValTok{5}\NormalTok{]}\SpecialCharTok{*}\NormalTok{X2)}\SpecialCharTok{/}\NormalTok{sigma2))  }
\NormalTok{\}}

\CommentTok{\# Linear variance function: |r| = gam1 + gam2*Yhat. 각 r이 variance{-}covariance matrix의 diagonal term이 된다. }
\CommentTok{\# For linear variance function, we can consider absolute residuals,}
\CommentTok{\# instead of squared residuals.}
\CommentTok{\# gam.hat = (Z\^{}T W Z)\^{}({-}1) Z\^{}T W |r|.}

\NormalTok{beta.new }\OtherTok{=}\NormalTok{ ml1}\SpecialCharTok{$}\NormalTok{par      }\CommentTok{\# initial parameter.}
\NormalTok{W }\OtherTok{=} \FunctionTok{diag}\NormalTok{(}\FunctionTok{rep}\NormalTok{(}\DecValTok{1}\NormalTok{,}\FunctionTok{length}\NormalTok{(Y))) }\CommentTok{\# W에 대한 정보는 따로 없으므로 등분산 가정 }
\NormalTok{mdif }\OtherTok{=} \DecValTok{100000}

\ControlFlowTok{while}\NormalTok{(mdif }\SpecialCharTok{\textgreater{}} \FloatTok{0.000001}\NormalTok{)}
\NormalTok{\{}
\NormalTok{  Yhat }\OtherTok{=} \FunctionTok{f}\NormalTok{(beta.new,X)}
\NormalTok{  r }\OtherTok{=}\NormalTok{ Y }\SpecialCharTok{{-}}\NormalTok{ Yhat }\CommentTok{\#보통 squared residuals를 사용하는데 이 예제에서는 absolute residuals를 사용하였다. }
\NormalTok{  Z }\OtherTok{=} \FunctionTok{cbind}\NormalTok{(}\DecValTok{1}\NormalTok{,Yhat)}
\NormalTok{  gam.hat }\OtherTok{=} \FunctionTok{solve}\NormalTok{(}\FunctionTok{t}\NormalTok{(Z) }\SpecialCharTok{\%*\%}\NormalTok{ W }\SpecialCharTok{\%*\%}\NormalTok{ Z) }\SpecialCharTok{\%*\%} \FunctionTok{t}\NormalTok{(Z) }\SpecialCharTok{\%*\%}\NormalTok{ W }\SpecialCharTok{\%*\%} \FunctionTok{abs}\NormalTok{(r)}
\NormalTok{  sigma }\OtherTok{=}\NormalTok{ Z }\SpecialCharTok{\%*\%}\NormalTok{ gam.hat}
\NormalTok{  S }\OtherTok{=} \FunctionTok{diag}\NormalTok{(}\FunctionTok{as.vector}\NormalTok{(sigma}\SpecialCharTok{\^{}}\DecValTok{2}\NormalTok{))}\CommentTok{\#objective mean function에 사용될 sigma}
  
  \ControlFlowTok{if}\NormalTok{ (}\FunctionTok{is.non.singular.matrix}\NormalTok{(S)) W }\OtherTok{=} \FunctionTok{solve}\NormalTok{(S)}
  \ControlFlowTok{else}\NormalTok{ W }\OtherTok{=} \FunctionTok{solve}\NormalTok{(S }\SpecialCharTok{+} \FloatTok{0.000000001}\SpecialCharTok{*}\FunctionTok{diag}\NormalTok{(}\FunctionTok{rep}\NormalTok{(}\DecValTok{1}\NormalTok{,}\FunctionTok{nrow}\NormalTok{(S)))) }\CommentTok{\#variance function구할 때의 weights}

\NormalTok{  ml2 }\OtherTok{=} \FunctionTok{optim}\NormalTok{(beta.new, obj.mean, }\AttributeTok{gr=}\NormalTok{gr.mean,}\AttributeTok{method=}\StringTok{\textquotesingle{}BFGS\textquotesingle{}}\NormalTok{, }\AttributeTok{Y=}\NormalTok{Y, }\AttributeTok{X=}\NormalTok{X, }\AttributeTok{S=}\NormalTok{S)}
\NormalTok{  beta.old }\OtherTok{=}\NormalTok{ beta.new}
\NormalTok{  beta.new }\OtherTok{=}\NormalTok{ ml2}\SpecialCharTok{$}\NormalTok{par}
\NormalTok{  mdif }\OtherTok{=} \FunctionTok{max}\NormalTok{(}\FunctionTok{abs}\NormalTok{(beta.new }\SpecialCharTok{{-}}\NormalTok{ beta.old))}
\NormalTok{\}}

\NormalTok{beta.new}
\end{Highlighting}
\end{Shaded}

\begin{verbatim}
## [1] -1.08510428  0.30750346 -0.02149807  0.33393746  0.02685590
\end{verbatim}

\begin{Shaded}
\begin{Highlighting}[]
\NormalTok{Yhat }\OtherTok{=} \FunctionTok{f}\NormalTok{(beta.new,X)}
\NormalTok{sigma }\OtherTok{=}\NormalTok{ Z }\SpecialCharTok{\%*\%}\NormalTok{ gam.hat}
\NormalTok{r }\OtherTok{=}\NormalTok{ (Y }\SpecialCharTok{{-}}\NormalTok{ Yhat)}\SpecialCharTok{/}\NormalTok{sigma}

\CommentTok{\# Residual plot}
\FunctionTok{plot}\NormalTok{(Yhat,r,}\AttributeTok{ylim=}\FunctionTok{c}\NormalTok{(}\SpecialCharTok{{-}}\DecValTok{4}\NormalTok{,}\DecValTok{4}\NormalTok{))}
\FunctionTok{lines}\NormalTok{(}\FunctionTok{c}\NormalTok{(}\DecValTok{0}\NormalTok{,}\DecValTok{10}\NormalTok{),}\FunctionTok{c}\NormalTok{(}\DecValTok{0}\NormalTok{,}\DecValTok{0}\NormalTok{),}\AttributeTok{col=}\StringTok{\textquotesingle{}red\textquotesingle{}}\NormalTok{)}
\CommentTok{\#random error의 형태를 띔.}
\CommentTok{\# 최종 모델은 이 variance function을 포함한 모델이다.}

\DocumentationTok{\#\#\#\#\# Linear regression with nonconstant variance \#\#\#\#\#}
\CommentTok{\# lmvar: }
\CommentTok{\# Linear mean function}
\CommentTok{\# Linear variance function: log(sigma) = X*beta}
\CommentTok{\# install.packages(\textquotesingle{}lmvar\textquotesingle{})}
\FunctionTok{library}\NormalTok{(lmvar)}

\NormalTok{X\_s }\OtherTok{=} \FunctionTok{cbind}\NormalTok{(attenu}\SpecialCharTok{$}\NormalTok{mag, attenu}\SpecialCharTok{$}\NormalTok{dist)}
\FunctionTok{colnames}\NormalTok{(X\_s) }\OtherTok{=} \FunctionTok{c}\NormalTok{(}\StringTok{\textquotesingle{}mag\textquotesingle{}}\NormalTok{, }\StringTok{\textquotesingle{}dist\textquotesingle{}}\NormalTok{)}
\NormalTok{fit3 }\OtherTok{=} \FunctionTok{lmvar}\NormalTok{(attenu}\SpecialCharTok{$}\NormalTok{accel, X, X\_s)}
\end{Highlighting}
\end{Shaded}

\includegraphics{SM_Example1_files/figure-latex/unnamed-chunk-1-4.pdf}

\begin{Shaded}
\begin{Highlighting}[]
\FunctionTok{summary}\NormalTok{(fit3)}
\end{Highlighting}
\end{Shaded}

\begin{verbatim}
## Call: 
##  lmvar(y = attenu$accel, X_mu = X, X_sigma = X_s)
## 
## Number of observations:  182 
## Degrees of freedom    :  6 
## 
## Z-scores: 
##     Min      1Q  Median      3Q     Max 
## -2.4255  0.2383  0.6385  1.0469  2.5417 
## 
## Coefficients:
##                  Estimate  Std. Error  z value  Pr(>|z|)    
## (Intercept)    1.4106e-02  1.2999e-02   1.0852    0.2779    
## mag            2.4043e-03  1.7962e-03   1.3386    0.1807    
## dist          -7.6248e-05  6.6991e-06 -11.3819 < 2.2e-16 ***
## (Intercept_s) -4.1948e+00  4.9252e-01  -8.5171 < 2.2e-16 ***
## mag_s          4.7234e-01  8.3854e-02   5.6328 1.773e-08 ***
## dist_s        -2.1555e-02  9.7305e-04 -22.1514 < 2.2e-16 ***
## ---
## Signif. codes:  0 '***' 0.001 '**' 0.01 '*' 0.05 '.' 0.1 ' ' 1
## 
## Standard deviations: 
##    Min     1Q Median     3Q    Max 
## 0.0002 0.0843 0.1310 0.2018 0.3213 
## 
## Comparison to model with constant variance (i.e. classical linear model)
## Log likelihood-ratio: 37.74527 
## Additional degrees of freedom: 2 
## p-value for difference in deviance: 4.05e-17
\end{verbatim}

\begin{Shaded}
\begin{Highlighting}[]
\NormalTok{ms }\OtherTok{=} \FunctionTok{predict}\NormalTok{(fit3, }\AttributeTok{X\_mu=}\NormalTok{X, }\AttributeTok{X\_sigma=}\NormalTok{X\_s)}
\NormalTok{r1 }\OtherTok{=}\NormalTok{ (Y }\SpecialCharTok{{-}}\NormalTok{ ms[,}\DecValTok{1}\NormalTok{])}\SpecialCharTok{/}\NormalTok{ms[,}\DecValTok{2}\NormalTok{]}
\FunctionTok{plot}\NormalTok{(ms[,}\DecValTok{1}\NormalTok{],r1)}
\FunctionTok{lines}\NormalTok{(}\FunctionTok{c}\NormalTok{(}\SpecialCharTok{{-}}\DecValTok{10}\NormalTok{,}\DecValTok{10}\NormalTok{),}\FunctionTok{c}\NormalTok{(}\DecValTok{0}\NormalTok{,}\DecValTok{0}\NormalTok{),}\AttributeTok{col=}\StringTok{\textquotesingle{}red\textquotesingle{}}\NormalTok{)}
\end{Highlighting}
\end{Shaded}

\includegraphics{SM_Example1_files/figure-latex/unnamed-chunk-1-5.pdf}

\begin{Shaded}
\begin{Highlighting}[]
\CommentTok{\#5주차 1차시 끝}
\CommentTok{\#\textasciitilde{}\textasciitilde{}\textasciitilde{}\textasciitilde{}\textasciitilde{}\textasciitilde{}\textasciitilde{}\textasciitilde{}\textasciitilde{}\textasciitilde{}\textasciitilde{}\textasciitilde{}\textasciitilde{}\textasciitilde{}\textasciitilde{}\textasciitilde{}\textasciitilde{}\textasciitilde{}\textasciitilde{}\textasciitilde{}\textasciitilde{}\textasciitilde{}\textasciitilde{}\textasciitilde{}\textasciitilde{}\textasciitilde{}\textasciitilde{}\textasciitilde{}\textasciitilde{}\textasciitilde{}\textasciitilde{}\textasciitilde{}\textasciitilde{}\textasciitilde{}\textasciitilde{}\textasciitilde{}\textasciitilde{}\textasciitilde{}\textasciitilde{}\textasciitilde{}\textasciitilde{}\textasciitilde{}\textasciitilde{}\textasciitilde{}\textasciitilde{}\textasciitilde{}\textasciitilde{}\textasciitilde{}\textasciitilde{}\textasciitilde{}\textasciitilde{}\textasciitilde{}\textasciitilde{}\textasciitilde{}\textasciitilde{}\textasciitilde{}\textasciitilde{}\textasciitilde{}\textasciitilde{}\textasciitilde{}\textasciitilde{}\textasciitilde{}\textasciitilde{}\textasciitilde{}\textasciitilde{}\textasciitilde{}\textasciitilde{}\textasciitilde{}\textasciitilde{}\textasciitilde{}\textasciitilde{}\textasciitilde{}\textasciitilde{}\textasciitilde{}\textasciitilde{}\textasciitilde{}\textasciitilde{}\textasciitilde{}}


\DocumentationTok{\#\#\#\#\#\#\#\#\#\#\#\#\#\#\#\#\#\#\#\#\#\#\#\#\#\#\#\#\#\#\#\#\#\#\#\#\#\#\#\#\#\#\#\#\#\#\#\#\#\#\#\#\#\#}
\CommentTok{\# Modelling Example 2:  Model with time correlations \#}
\DocumentationTok{\#\#\#\#\#\#\#\#\#\#\#\#\#\#\#\#\#\#\#\#\#\#\#\#\#\#\#\#\#\#\#\#\#\#\#\#\#\#\#\#\#\#\#\#\#\#\#\#\#\#\#\#\#\#}

\NormalTok{tsdat }\OtherTok{=} \FunctionTok{read.table}\NormalTok{(}\StringTok{\textquotesingle{}tsdat.txt\textquotesingle{}}\NormalTok{,}\AttributeTok{header=}\NormalTok{T)}

\NormalTok{fit }\OtherTok{=} \FunctionTok{lm}\NormalTok{(y }\SpecialCharTok{\textasciitilde{}}\NormalTok{ x, }\AttributeTok{data=}\NormalTok{tsdat)}
\FunctionTok{summary}\NormalTok{(fit)}
\end{Highlighting}
\end{Shaded}

\begin{verbatim}
## 
## Call:
## lm(formula = y ~ x, data = tsdat)
## 
## Residuals:
##     Min      1Q  Median      3Q     Max 
## -61.998 -16.403   2.365  12.663  57.931 
## 
## Coefficients:
##              Estimate Std. Error t value Pr(>|t|)    
## (Intercept) 3.600e+04  7.152e+01 503.377  < 2e-16 ***
## x           1.610e+00  2.023e-01   7.957 1.56e-11 ***
## ---
## Signif. codes:  0 '***' 0.001 '**' 0.01 '*' 0.05 '.' 0.1 ' ' 1
## 
## Residual standard error: 25.83 on 74 degrees of freedom
## Multiple R-squared:  0.4611, Adjusted R-squared:  0.4538 
## F-statistic: 63.31 on 1 and 74 DF,  p-value: 1.565e-11
\end{verbatim}

\begin{Shaded}
\begin{Highlighting}[]
\FunctionTok{par}\NormalTok{(}\AttributeTok{mfrow=}\FunctionTok{c}\NormalTok{(}\DecValTok{2}\NormalTok{,}\DecValTok{2}\NormalTok{))}
\FunctionTok{plot}\NormalTok{(fit)}
\end{Highlighting}
\end{Shaded}

\includegraphics{SM_Example1_files/figure-latex/unnamed-chunk-1-6.pdf}

\begin{Shaded}
\begin{Highlighting}[]
\CommentTok{\# 등분산, normality를 만족하는 것으로 보인다. }
\CommentTok{\#하지만 time series이기 때문에 linear test를 해봐야 한다. Durbin{-}Watson Test}

\CommentTok{\# Durbin{-}Watson test}
\CommentTok{\# install.packages(\textquotesingle{}lmtest\textquotesingle{})}
\FunctionTok{library}\NormalTok{(lmtest)}
\end{Highlighting}
\end{Shaded}

\begin{verbatim}
## Loading required package: zoo
\end{verbatim}

\begin{verbatim}
## 
## Attaching package: 'zoo'
\end{verbatim}

\begin{verbatim}
## The following objects are masked from 'package:base':
## 
##     as.Date, as.Date.numeric
\end{verbatim}

\begin{Shaded}
\begin{Highlighting}[]
\FunctionTok{dwtest}\NormalTok{(fit) }\CommentTok{\# time correlation이 있음이 명확하다. }
\end{Highlighting}
\end{Shaded}

\begin{verbatim}
## 
##  Durbin-Watson test
## 
## data:  fit
## DW = 0.69523, p-value = 4.72e-11
## alternative hypothesis: true autocorrelation is greater than 0
\end{verbatim}

\begin{Shaded}
\begin{Highlighting}[]
\CommentTok{\# Check ACF \& PACF}
\CommentTok{\# install.packages(\textquotesingle{}astsa\textquotesingle{})}
\FunctionTok{library}\NormalTok{(astsa)}

\CommentTok{\# AR(p): ACF: Exponentially decreasing; PACF: Non{-}zero values at first p lags.}
\CommentTok{\# MA(q): ACF: Non{-}zero values at first q lags; PACF: Exponentially decreasing.}
\CommentTok{\# ARMA(p,q): ACF: Similar to ACF of AR(p); PACF: Similar to PACF of MA(q).}

\FunctionTok{acf2}\NormalTok{(}\FunctionTok{residuals}\NormalTok{(fit)) }\CommentTok{\# acf는 exponentially decreasing하다는 것을 알 수 있고, pacf는 1 이후로 correlation이 0에 가깝다는 것을 알 수 있다. }
\end{Highlighting}
\end{Shaded}

\includegraphics{SM_Example1_files/figure-latex/unnamed-chunk-1-7.pdf}

\begin{verbatim}
##      [,1]  [,2]  [,3]  [,4]  [,5]  [,6]  [,7]  [,8] [,9] [,10] [,11] [,12]
## ACF  0.64  0.36  0.08 -0.07 -0.12 -0.11 -0.11 -0.04 0.04  0.02 -0.05 -0.08
## PACF 0.64 -0.07 -0.20 -0.02  0.00 -0.01 -0.06  0.06 0.09 -0.13 -0.12  0.05
##      [,13] [,14] [,15] [,16] [,17] [,18] [,19]
## ACF  -0.09  -0.1 -0.03 -0.01  0.04  0.12  0.10
## PACF  0.02  -0.1  0.09 -0.01  0.02  0.10 -0.08
\end{verbatim}

\begin{Shaded}
\begin{Highlighting}[]
\CommentTok{\# lag = 1 이면 어지간한 time correlation을 잡을 수 있겠다는 추정이 가능 =\textgreater{} AR(1)모델 사용}

\NormalTok{ar1 }\OtherTok{=} \FunctionTok{sarima}\NormalTok{ (}\FunctionTok{residuals}\NormalTok{(fit), }\AttributeTok{p =} \DecValTok{1}\NormalTok{,}\AttributeTok{d =} \DecValTok{0}\NormalTok{,}\AttributeTok{q =} \DecValTok{0}\NormalTok{, }\AttributeTok{no.constant=}\NormalTok{T)   }\CommentTok{\#AR(1)}
\end{Highlighting}
\end{Shaded}

\begin{verbatim}
## initial  value 3.229004 
## iter   2 value 2.958676
## iter   3 value 2.958511
## iter   4 value 2.958366
## iter   4 value 2.958366
## final  value 2.958366 
## converged
## initial  value 2.971232 
## iter   2 value 2.971136
## iter   3 value 2.971118
## iter   3 value 2.971118
## iter   3 value 2.971118
## final  value 2.971118 
## converged
\end{verbatim}

\includegraphics{SM_Example1_files/figure-latex/unnamed-chunk-1-8.pdf}

\begin{Shaded}
\begin{Highlighting}[]
\CommentTok{\# p = AR의 차수, q = MA의 차수, no.constant = T : mean = 0 가정, d는 어차피 시험에 나오지도 않을거임}
\NormalTok{ar1}\SpecialCharTok{$}\NormalTok{fit  }\CommentTok{\#time dependency가 사라짐. }
\end{Highlighting}
\end{Shaded}

\begin{verbatim}
## 
## Call:
## arima(x = xdata, order = c(p, d, q), seasonal = list(order = c(P, D, Q), period = S), 
##     xreg = xmean, include.mean = FALSE, transform.pars = trans, fixed = fixed, 
##     optim.control = list(trace = trc, REPORT = 1, reltol = tol))
## 
## Coefficients:
##          ar1
##       0.6488
## s.e.  0.0875
## 
## sigma^2 estimated as 378.1:  log likelihood = -333.64,  aic = 671.29
\end{verbatim}

\begin{Shaded}
\begin{Highlighting}[]
\CommentTok{\# 그럼 이걸 이용해서 어떻게 beta를 추정할 것인가의 문제를 풀면 된다. }


\CommentTok{\# MLE: Multivariate normal distribution}
\NormalTok{X }\OtherTok{=} \FunctionTok{cbind}\NormalTok{(}\DecValTok{1}\NormalTok{,tsdat}\SpecialCharTok{$}\NormalTok{x)}
\NormalTok{Y }\OtherTok{=}\NormalTok{ tsdat}\SpecialCharTok{$}\NormalTok{y}
\NormalTok{n }\OtherTok{=} \FunctionTok{length}\NormalTok{(Y)}
\NormalTok{S }\OtherTok{=} \FunctionTok{diag}\NormalTok{(}\FunctionTok{rep}\NormalTok{(}\DecValTok{1}\NormalTok{,n))    }\CommentTok{\# initial covariance matrix}

\NormalTok{mdif }\OtherTok{=} \DecValTok{1000000}
\NormalTok{beta.old }\OtherTok{=} \FunctionTok{rep}\NormalTok{(}\DecValTok{100000}\NormalTok{,}\DecValTok{2}\NormalTok{) }\CommentTok{\# Y의 scale에 따라 Y보다는 어느 정도 크게 잡아야 하는 것 같다. }

\ControlFlowTok{while}\NormalTok{(mdif }\SpecialCharTok{\textgreater{}} \FloatTok{0.0000001}\NormalTok{)}
\NormalTok{\{}
\NormalTok{  beta.new }\OtherTok{=} \FunctionTok{as.vector}\NormalTok{(}\FunctionTok{solve}\NormalTok{(}\FunctionTok{t}\NormalTok{(X) }\SpecialCharTok{\%*\%} \FunctionTok{solve}\NormalTok{(S) }\SpecialCharTok{\%*\%}\NormalTok{ X) }\SpecialCharTok{\%*\%}\FunctionTok{t}\NormalTok{(X) }\SpecialCharTok{\%*\%} \FunctionTok{solve}\NormalTok{(S) }\SpecialCharTok{\%*\%}\NormalTok{ Y) }\CommentTok{\# Weighted Least Squares }
\NormalTok{  r }\OtherTok{=} \FunctionTok{as.vector}\NormalTok{(Y }\SpecialCharTok{{-}}\NormalTok{ (X }\SpecialCharTok{\%*\%}\NormalTok{ beta.new))}
\NormalTok{  ar1 }\OtherTok{=} \FunctionTok{sarima}\NormalTok{ (r, }\DecValTok{1}\NormalTok{,}\DecValTok{0}\NormalTok{,}\DecValTok{0}\NormalTok{, }\AttributeTok{no.constant=}\NormalTok{T, }\AttributeTok{details=}\NormalTok{F)}
\NormalTok{  alpha }\OtherTok{=}\NormalTok{ ar1}\SpecialCharTok{$}\NormalTok{fit}\SpecialCharTok{$}\NormalTok{coef }\CommentTok{\# AR(1)모델의 coefficient}
\NormalTok{  sigma2 }\OtherTok{=}\NormalTok{ ar1}\SpecialCharTok{$}\NormalTok{fit}\SpecialCharTok{$}\NormalTok{sigma2 }\CommentTok{\# AR(1)모델의 sigma squared}
  
\NormalTok{  mdif }\OtherTok{=} \FunctionTok{max}\NormalTok{(}\FunctionTok{abs}\NormalTok{(beta.new }\SpecialCharTok{{-}}\NormalTok{ beta.old))}
\NormalTok{  beta.old }\OtherTok{=}\NormalTok{ beta.new}
  \CommentTok{\# Construct covariance matrix}
\NormalTok{  S }\OtherTok{=} \FunctionTok{matrix}\NormalTok{(}\AttributeTok{nrow=}\NormalTok{n,}\AttributeTok{ncol=}\NormalTok{n)}
  \ControlFlowTok{for}\NormalTok{ (i }\ControlFlowTok{in} \DecValTok{1}\SpecialCharTok{:}\NormalTok{n)}
\NormalTok{  \{}
    \ControlFlowTok{for}\NormalTok{ (j }\ControlFlowTok{in} \DecValTok{1}\SpecialCharTok{:}\NormalTok{n)}
\NormalTok{    \{}
      \ControlFlowTok{if}\NormalTok{ (i }\SpecialCharTok{==}\NormalTok{ j) S[i,j] }\OtherTok{=} \DecValTok{1}
      \ControlFlowTok{if}\NormalTok{ (i }\SpecialCharTok{!=}\NormalTok{ j) S[i,j] }\OtherTok{=}\NormalTok{ alpha}\SpecialCharTok{\^{}}\NormalTok{(}\FunctionTok{abs}\NormalTok{(i}\SpecialCharTok{{-}}\NormalTok{j))}
\NormalTok{    \}}
\NormalTok{  \}}
\NormalTok{  S }\OtherTok{=}\NormalTok{ (sigma2 }\SpecialCharTok{/}\NormalTok{ (}\DecValTok{1}\SpecialCharTok{{-}}\NormalTok{alpha}\SpecialCharTok{\^{}}\DecValTok{2}\NormalTok{)) }\SpecialCharTok{*}\NormalTok{ S }\CommentTok{\# updating covariance matrix}
\NormalTok{\}}

\FunctionTok{round}\NormalTok{(beta.new,}\DecValTok{4}\NormalTok{)}
\end{Highlighting}
\end{Shaded}

\begin{verbatim}
## [1] 35986.2865     1.6521
\end{verbatim}

\begin{Shaded}
\begin{Highlighting}[]
\CommentTok{\# MLE: Product of conditional distribution (Approximation)}
\CommentTok{\# Y\_t | Y\_t{-}1 \textasciitilde{} N(X\_t*beta + alpha*epsilon\_t{-}1, sigma\^{}2)}

\NormalTok{fit }\OtherTok{=} \FunctionTok{lm}\NormalTok{(y }\SpecialCharTok{\textasciitilde{}}\NormalTok{ x, }\AttributeTok{data=}\NormalTok{tsdat)}

\NormalTok{Yt }\OtherTok{=}\NormalTok{ tsdat}\SpecialCharTok{$}\NormalTok{y[}\DecValTok{2}\SpecialCharTok{:}\NormalTok{n]}
\NormalTok{Xt }\OtherTok{=}\NormalTok{ tsdat}\SpecialCharTok{$}\NormalTok{x[}\DecValTok{2}\SpecialCharTok{:}\NormalTok{n]}
\NormalTok{et }\OtherTok{=} \FunctionTok{residuals}\NormalTok{(fit)[}\DecValTok{1}\SpecialCharTok{:}\NormalTok{(n}\DecValTok{{-}1}\NormalTok{)] }\CommentTok{\# Y\_t | Y\_t{-}1 \textasciitilde{} N(X\_t*beta + alpha*epsilon\_t{-}1, sigma\^{}2)이니까 n{-}1까지 }
\NormalTok{mdif }\OtherTok{=} \DecValTok{10000}
\NormalTok{b.old }\OtherTok{=} \FunctionTok{rep}\NormalTok{(}\DecValTok{0}\NormalTok{,}\DecValTok{3}\NormalTok{)}
 
\ControlFlowTok{while}\NormalTok{(mdif }\SpecialCharTok{\textgreater{}} \FloatTok{0.0000001}\NormalTok{)}
\NormalTok{\{}
\NormalTok{  fit.temp }\OtherTok{=} \FunctionTok{lm}\NormalTok{(Yt }\SpecialCharTok{\textasciitilde{}}\NormalTok{ Xt }\SpecialCharTok{+}\NormalTok{ et)}
\NormalTok{  b.new }\OtherTok{=}\NormalTok{ fit.temp}\SpecialCharTok{$}\NormalTok{coefficient}
\NormalTok{  mdif }\OtherTok{=} \FunctionTok{max}\NormalTok{(}\FunctionTok{abs}\NormalTok{(b.new[}\DecValTok{1}\SpecialCharTok{:}\DecValTok{2}\NormalTok{] }\SpecialCharTok{{-}}\NormalTok{ b.old[}\DecValTok{1}\SpecialCharTok{:}\DecValTok{2}\NormalTok{]))}
  
\NormalTok{  et }\OtherTok{=}\NormalTok{ (Y }\SpecialCharTok{{-}}\NormalTok{ X }\SpecialCharTok{\%*\%}\NormalTok{ b.new[}\DecValTok{1}\SpecialCharTok{:}\DecValTok{2}\NormalTok{])[}\DecValTok{1}\SpecialCharTok{:}\NormalTok{(n}\DecValTok{{-}1}\NormalTok{)]}
\NormalTok{  b.old }\OtherTok{=}\NormalTok{ b.new}
\NormalTok{\}}

\FunctionTok{round}\NormalTok{(b.new,}\DecValTok{4}\NormalTok{)}
\end{Highlighting}
\end{Shaded}

\begin{verbatim}
## (Intercept)          Xt          et 
##  35999.7941      1.6205      0.6379
\end{verbatim}

\begin{Shaded}
\begin{Highlighting}[]
\CommentTok{\# Built{-}in function }
\CommentTok{\# cochrane.orcutt =\textgreater{} f: linear model, error: AR(p) process.}
\CommentTok{\# install.packages("orcutt")}
\FunctionTok{library}\NormalTok{(orcutt)}

\NormalTok{fit }\OtherTok{=} \FunctionTok{lm}\NormalTok{(y }\SpecialCharTok{\textasciitilde{}}\NormalTok{ x, }\AttributeTok{data=}\NormalTok{tsdat)}
\FunctionTok{cochrane.orcutt}\NormalTok{(fit)}
\end{Highlighting}
\end{Shaded}

\begin{verbatim}
## Cochrane-orcutt estimation for first order autocorrelation 
##  
## Call:
## lm(formula = y ~ x, data = tsdat)
## 
##  number of interaction: 4
##  rho 0.63843
## 
## Durbin-Watson statistic 
## (original):    0.69523 , p-value: 4.72e-11
## (transformed): 1.83115 , p-value: 2.515e-01
##  
##  coefficients: 
##  (Intercept)            x 
## 35994.746022     1.634731
\end{verbatim}

\begin{Shaded}
\begin{Highlighting}[]
\DocumentationTok{\#\#\#\#\#\#\#\#\#\#\#\#\#\#\#\#\#\#\#\#\#\#\#\#\#\#\#\#\#\#\#\#\#\#\#\#\#\#\#\#\#\#\#\#\#\#\#\#\#\#\#\#\#\#\#\#\#}
\CommentTok{\# Modelling Example 3:  Model with spatial correlations \#}
\DocumentationTok{\#\#\#\#\#\#\#\#\#\#\#\#\#\#\#\#\#\#\#\#\#\#\#\#\#\#\#\#\#\#\#\#\#\#\#\#\#\#\#\#\#\#\#\#\#\#\#\#\#\#\#\#\#\#\#\#\#}

\CommentTok{\# install.packages(\textquotesingle{}spdep\textquotesingle{})}
\FunctionTok{library}\NormalTok{(spdep)}
\end{Highlighting}
\end{Shaded}

\begin{verbatim}
## Loading required package: sp
\end{verbatim}

\begin{verbatim}
## Loading required package: spData
\end{verbatim}

\begin{verbatim}
## To access larger datasets in this package, install the spDataLarge
## package with: `install.packages('spDataLarge',
## repos='https://nowosad.github.io/drat/', type='source')`
\end{verbatim}

\begin{verbatim}
## Loading required package: sf
\end{verbatim}

\begin{verbatim}
## Linking to GEOS 3.9.1, GDAL 3.2.3, PROJ 7.2.1
\end{verbatim}

\begin{Shaded}
\begin{Highlighting}[]
\FunctionTok{data}\NormalTok{(oldcol)}

\NormalTok{?COL.OLD}
\CommentTok{\# \textquotesingle{}COL.nb\textquotesingle{} has the neighbors list.}

\CommentTok{\# 2{-}D Coordinates of observations}
\NormalTok{crds }\OtherTok{=} \FunctionTok{cbind}\NormalTok{(COL.OLD}\SpecialCharTok{$}\NormalTok{X, COL.OLD}\SpecialCharTok{$}\NormalTok{Y)  }

\CommentTok{\# Compute the maximum distance }
\NormalTok{mdist }\OtherTok{=} \FunctionTok{sqrt}\NormalTok{(}\FunctionTok{sum}\NormalTok{(}\FunctionTok{diff}\NormalTok{(}\FunctionTok{apply}\NormalTok{(crds, }\DecValTok{2}\NormalTok{, range))}\SpecialCharTok{\^{}}\DecValTok{2}\NormalTok{))   }

\CommentTok{\# All obs. between 0 and mdist are identified as neighborhoods.}
\NormalTok{dnb }\OtherTok{=} \FunctionTok{dnearneigh}\NormalTok{(crds, }\DecValTok{0}\NormalTok{, mdist)}

\CommentTok{\# Compute Euclidean distance between obs.}
\NormalTok{dists }\OtherTok{=} \FunctionTok{nbdists}\NormalTok{(dnb, crds)}

\CommentTok{\# Compute Power distance weight d\^{}({-}2)}
\NormalTok{glst }\OtherTok{=} \FunctionTok{lapply}\NormalTok{(dists, }\ControlFlowTok{function}\NormalTok{(d) d}\SpecialCharTok{\^{}}\NormalTok{(}\SpecialCharTok{{-}}\DecValTok{2}\NormalTok{))}

\CommentTok{\# Construct weight matrix with normalization }
\CommentTok{\# style=\textquotesingle{}C\textquotesingle{}: global normalization; \textquotesingle{}W\textquotesingle{}: row normalization}
\NormalTok{lw }\OtherTok{=} \FunctionTok{nb2listw}\NormalTok{(dnb, }\AttributeTok{glist=}\NormalTok{glst, }\AttributeTok{style=}\StringTok{\textquotesingle{}C\textquotesingle{}}\NormalTok{)}

\CommentTok{\# Spatial Autoregressive Model}
\NormalTok{fit }\OtherTok{=} \FunctionTok{lagsarlm}\NormalTok{(CRIME }\SpecialCharTok{\textasciitilde{}}\NormalTok{ HOVAL }\SpecialCharTok{+}\NormalTok{ INC, }\AttributeTok{data=}\NormalTok{COL.OLD, }\AttributeTok{listw=}\NormalTok{lw) }\CommentTok{\#listw : normalized weight matrix를 넣는 곳인 것 같은데}
\end{Highlighting}
\end{Shaded}

\begin{verbatim}
## Warning in lagsarlm(CRIME ~ HOVAL + INC, data = COL.OLD, listw = lw): install
## the spatialreg package
\end{verbatim}

\begin{verbatim}
## Warning: Function jacobianSetup moved to the spatialreg package
\end{verbatim}

\begin{verbatim}
## Warning: Function do_ldet moved to the spatialreg package

## Warning: Function do_ldet moved to the spatialreg package

## Warning: Function do_ldet moved to the spatialreg package

## Warning: Function do_ldet moved to the spatialreg package

## Warning: Function do_ldet moved to the spatialreg package

## Warning: Function do_ldet moved to the spatialreg package

## Warning: Function do_ldet moved to the spatialreg package

## Warning: Function do_ldet moved to the spatialreg package

## Warning: Function do_ldet moved to the spatialreg package

## Warning: Function do_ldet moved to the spatialreg package

## Warning: Function do_ldet moved to the spatialreg package

## Warning: Function do_ldet moved to the spatialreg package

## Warning: Function do_ldet moved to the spatialreg package
\end{verbatim}

\begin{Shaded}
\begin{Highlighting}[]
\FunctionTok{summary}\NormalTok{(fit)}
\end{Highlighting}
\end{Shaded}

\begin{verbatim}
## Warning in summary.sarlm(fit): install the spatialreg package
\end{verbatim}

\begin{verbatim}
## Warning in Wald1.sarlm(object): install the spatialreg package
\end{verbatim}

\begin{verbatim}
## Warning in logLik.sarlm(object): install the spatialreg package
\end{verbatim}

\begin{verbatim}
## Warning in residuals.sarlm(object): install the spatialreg package
\end{verbatim}

\begin{verbatim}
## Warning in LR1.sarlm(object): install the spatialreg package
\end{verbatim}

\begin{verbatim}
## Warning in logLik.sarlm(object): install the spatialreg package
\end{verbatim}

\begin{verbatim}
## Warning in residuals.sarlm(object): install the spatialreg package
\end{verbatim}

\begin{verbatim}
## Warning in print.summary.sarlm(x): install the spatialreg package
\end{verbatim}

\begin{verbatim}
## 
## Call:lagsarlm(formula = CRIME ~ HOVAL + INC, data = COL.OLD, listw = lw)
## 
## Residuals:
\end{verbatim}

\begin{verbatim}
## Warning in residuals.sarlm(x): install the spatialreg package
\end{verbatim}

\begin{verbatim}
##       Min        1Q    Median        3Q       Max 
## -25.93357  -4.98320  -0.50733   4.96160  25.17053 
## 
## Type: lag 
## Coefficients: (asymptotic standard errors) 
##             Estimate Std. Error z value  Pr(>|z|)
## (Intercept) 46.04321    5.01268  9.1853 < 2.2e-16
## HOVAL       -0.23848    0.07684 -3.1036  0.001912
## INC         -0.85776    0.27588 -3.1092  0.001876
## 
## Rho: 0.24802, LR test value: 25.318, p-value: 4.8614e-07
## Asymptotic standard error: 0.038854
##     z-value: 6.3834, p-value: 1.7316e-10
## Wald statistic: 40.748, p-value: 1.7316e-10
\end{verbatim}

\begin{verbatim}
## Warning in logLik.sarlm(x): install the spatialreg package
\end{verbatim}

\begin{verbatim}
## Warning in residuals.sarlm(object): install the spatialreg package
\end{verbatim}

\begin{verbatim}
## 
## Log likelihood: -174.7182 for lag model
## ML residual variance (sigma squared): 72.254, (sigma: 8.5002)
## Number of observations: 49 
## Number of parameters estimated: 5
\end{verbatim}

\begin{verbatim}
## Warning in logLik.sarlm(object): install the spatialreg package

## Warning in logLik.sarlm(object): install the spatialreg package
\end{verbatim}

\begin{verbatim}
## AIC: 359.44, (AIC for lm: 382.75)
## LM test for residual autocorrelation
## test value: 3.0704, p-value: 0.079732
\end{verbatim}

\begin{Shaded}
\begin{Highlighting}[]
\CommentTok{\#ƒ}
\CommentTok{\# install.packages(\textquotesingle{}spatialreg\textquotesingle{})}
\FunctionTok{library}\NormalTok{(spatialreg)}
\end{Highlighting}
\end{Shaded}

\begin{verbatim}
## Loading required package: Matrix
\end{verbatim}

\begin{verbatim}
## 
## Attaching package: 'spatialreg'
\end{verbatim}

\begin{verbatim}
## The following objects are masked from 'package:spdep':
## 
##     as_dgRMatrix_listw, as_dsCMatrix_I, as_dsCMatrix_IrW,
##     as_dsTMatrix_listw, as.spam.listw, can.be.simmed, cheb_setup,
##     create_WX, do_ldet, eigen_pre_setup, eigen_setup, eigenw,
##     errorsarlm, get.ClusterOption, get.coresOption, get.mcOption,
##     get.VerboseOption, get.ZeroPolicyOption, GMargminImage, GMerrorsar,
##     griffith_sone, gstsls, Hausman.test, impacts, intImpacts,
##     Jacobian_W, jacobianSetup, l_max, lagmess, lagsarlm, lextrB,
##     lextrS, lextrW, lmSLX, LU_prepermutate_setup, LU_setup,
##     Matrix_J_setup, Matrix_setup, mcdet_setup, MCMCsamp, ME, mom_calc,
##     mom_calc_int2, moments_setup, powerWeights, sacsarlm,
##     SE_classic_setup, SE_interp_setup, SE_whichMin_setup,
##     set.ClusterOption, set.coresOption, set.mcOption,
##     set.VerboseOption, set.ZeroPolicyOption, similar.listw, spam_setup,
##     spam_update_setup, SpatialFiltering, spautolm, spBreg_err,
##     spBreg_lag, spBreg_sac, stsls, subgraph_eigenw, trW
\end{verbatim}

\begin{Shaded}
\begin{Highlighting}[]
\CommentTok{\# Fitted values}
\FunctionTok{predict}\NormalTok{(fit)}
\end{Highlighting}
\end{Shaded}

\begin{verbatim}
## Warning in predict.sarlm(fit): install the spatialreg package
\end{verbatim}

\begin{verbatim}
## Warning in fitted.sarlm(object): install the spatialreg package
\end{verbatim}

\begin{verbatim}
## Warning in print.sarlm.pred(x): install the spatialreg package
\end{verbatim}

\begin{verbatim}
## Warning in as.data.frame.sarlm.pred(x): install the spatialreg package
\end{verbatim}

\begin{verbatim}
##            fit       trend    signal
## 1001 21.139363 17.20296170  3.936402
## 1002 40.669968 34.28553076  6.384437
## 1003 36.930992 27.46526296  9.465729
## 1004 26.111841 20.91954551  5.192295
## 1005 12.853113 10.10064078  2.752472
## 1006 30.589972 26.07286878  4.517103
## 1007 38.576178 30.85302139  7.723156
## 1008 29.365348 25.43790510  3.927443
## 1009 47.328746 33.28192310 14.046823
## 1010 15.294461 11.39006196  3.904399
## 1011 31.907171 27.68239491  4.224776
## 1012 20.555831 16.51398582  4.041845
## 1013 23.612790 20.18482280  3.427967
## 1014 19.858166 16.95497060  2.903196
## 1015 21.665400 19.65330644  2.012094
## 1016  9.351186  5.67989040  3.671296
## 1017  6.191663  0.01215488  6.179508
## 1018 24.995203 18.41466270  6.580541
## 1019 38.688243 28.73801845  9.950225
## 1020 48.187784 34.26709989 13.920684
## 1021 40.752897 28.62840730 12.124490
## 1022 46.378557 31.90784522 14.470712
## 1023 40.556323 28.61294299 11.943380
## 1024 44.986360 31.10259763 13.883762
## 1025 32.462009 23.50569891  8.956310
## 1026 27.159847 21.38320752  5.776639
## 1027 38.591980 29.51925768  9.072722
## 1028 33.964208 27.30503607  6.659172
## 1029 41.812923 28.43747184 13.375451
## 1030 33.494142 21.12057693 12.373565
## 1031 48.738583 33.85540719 14.883176
## 1032 49.977939 34.51692950 15.461009
## 1033 48.587360 30.84643849 17.740921
## 1034 43.721512 28.74942043 14.972091
## 1035 34.260331 21.14507441 13.115257
## 1036 54.095323 35.01938745 19.075935
## 1037 65.956366 32.67870610 33.277660
## 1038 61.996250 34.94028272 27.055968
## 1039 55.430501 27.90790108 27.522600
## 1040 57.398367 27.26661378 30.131753
## 1041 58.046396 28.78030320 29.266092
## 1042 46.695348 20.42492885 26.270419
## 1043 37.606926 32.14015929  5.466767
## 1044 29.641722 26.43668716  3.205035
## 1045 26.839825 23.92912443  2.910701
## 1046 24.510130 21.31286074  3.197269
## 1047 22.888676 20.75061674  2.138059
## 1048 14.657706 12.17739523  2.480311
## 1049 16.230448 13.27789783  2.952550
\end{verbatim}

\begin{Shaded}
\begin{Highlighting}[]
\DocumentationTok{\#\#\#\#\#\#\#\#\#\#\#\#\#\#\#\#\#\#\#\#\#\#\#\#\#\#\#\#\#\#\#\#\#\#\#\#\#\#\#\#\#\#\#\#\#\#\#\#\#\#\#}
\CommentTok{\# Modelling Example 4:  Generalized Linear Models \#}
\DocumentationTok{\#\#\#\#\#\#\#\#\#\#\#\#\#\#\#\#\#\#\#\#\#\#\#\#\#\#\#\#\#\#\#\#\#\#\#\#\#\#\#\#\#\#\#\#\#\#\#\#\#\#\#}

\DocumentationTok{\#\#\#\#\#\#\#\#\#\# Cumulative logit model \#\#\#\#\#\#\#\#\#\#}
\CommentTok{\# install.packages(\textquotesingle{}ordinal\textquotesingle{})}
\FunctionTok{library}\NormalTok{(ordinal)}

\NormalTok{?wine}

\NormalTok{?clm}

\NormalTok{fit }\OtherTok{=} \FunctionTok{clm}\NormalTok{(rating }\SpecialCharTok{\textasciitilde{}}\NormalTok{ temp }\SpecialCharTok{+}\NormalTok{ contact, }\AttributeTok{data=}\NormalTok{wine, }\AttributeTok{link =} \StringTok{\textquotesingle{}logit\textquotesingle{}}\NormalTok{)}
\FunctionTok{summary}\NormalTok{(fit)}
\end{Highlighting}
\end{Shaded}

\begin{verbatim}
## formula: rating ~ temp + contact
## data:    wine
## 
##  link  threshold nobs logLik AIC    niter max.grad cond.H 
##  logit flexible  72   -86.49 184.98 6(0)  4.01e-12 2.7e+01
## 
## Coefficients:
##            Estimate Std. Error z value Pr(>|z|)    
## tempwarm     2.5031     0.5287   4.735 2.19e-06 ***
## contactyes   1.5278     0.4766   3.205  0.00135 ** 
## ---
## Signif. codes:  0 '***' 0.001 '**' 0.01 '*' 0.05 '.' 0.1 ' ' 1
## 
## Threshold coefficients:
##     Estimate Std. Error z value
## 1|2  -1.3444     0.5171  -2.600
## 2|3   1.2508     0.4379   2.857
## 3|4   3.4669     0.5978   5.800
## 4|5   5.0064     0.7309   6.850
\end{verbatim}

\begin{Shaded}
\begin{Highlighting}[]
\DocumentationTok{\#\#\#\#\#\#\#\#\#\# Poisson regression model \#\#\#\#\#\#\#\#\#\#}

\CommentTok{\# For data}
\CommentTok{\# install.packages(\textquotesingle{}lme4\textquotesingle{})}
\FunctionTok{library}\NormalTok{(lme4)}
\end{Highlighting}
\end{Shaded}

\begin{verbatim}
## 
## Attaching package: 'lme4'
\end{verbatim}

\begin{verbatim}
## The following objects are masked from 'package:ordinal':
## 
##     ranef, VarCorr
\end{verbatim}

\begin{Shaded}
\begin{Highlighting}[]
\FunctionTok{data}\NormalTok{(grouseticks)}

\NormalTok{?grouseticks}

\FunctionTok{head}\NormalTok{(grouseticks)}
\end{Highlighting}
\end{Shaded}

\begin{verbatim}
##   INDEX TICKS BROOD HEIGHT YEAR LOCATION   cHEIGHT
## 1     1     0   501    465   95       32  2.759305
## 2     2     0   501    465   95       32  2.759305
## 3     3     0   502    472   95       36  9.759305
## 4     4     0   503    475   95       37 12.759305
## 5     5     0   503    475   95       37 12.759305
## 6     6     3   503    475   95       37 12.759305
\end{verbatim}

\begin{Shaded}
\begin{Highlighting}[]
\FunctionTok{hist}\NormalTok{(grouseticks}\SpecialCharTok{$}\NormalTok{TICKS,}\AttributeTok{breaks=}\DecValTok{0}\SpecialCharTok{:}\DecValTok{90}\NormalTok{)}

\NormalTok{fit }\OtherTok{=} \FunctionTok{glm}\NormalTok{(TICKS }\SpecialCharTok{\textasciitilde{}}\NormalTok{ HEIGHT}\SpecialCharTok{*}\NormalTok{YEAR, }\AttributeTok{data =}\NormalTok{ grouseticks, }\AttributeTok{family=}\NormalTok{poisson)}
\FunctionTok{summary}\NormalTok{(fit)}
\end{Highlighting}
\end{Shaded}

\begin{verbatim}
## 
## Call:
## glm(formula = TICKS ~ HEIGHT * YEAR, family = poisson, data = grouseticks)
## 
## Deviance Residuals: 
##     Min       1Q   Median       3Q      Max  
## -6.0993  -1.7956  -0.8414   0.6453  14.1356  
## 
## Coefficients:
##                 Estimate Std. Error z value Pr(>|z|)    
## (Intercept)    27.454732   1.084156   25.32   <2e-16 ***
## HEIGHT         -0.058198   0.002539  -22.92   <2e-16 ***
## YEAR96        -18.994362   1.140285  -16.66   <2e-16 ***
## YEAR97        -19.247450   1.565774  -12.29   <2e-16 ***
## HEIGHT:YEAR96   0.044693   0.002662   16.79   <2e-16 ***
## HEIGHT:YEAR97   0.040453   0.003590   11.27   <2e-16 ***
## ---
## Signif. codes:  0 '***' 0.001 '**' 0.01 '*' 0.05 '.' 0.1 ' ' 1
## 
## (Dispersion parameter for poisson family taken to be 1)
## 
##     Null deviance: 5847.5  on 402  degrees of freedom
## Residual deviance: 3009.0  on 397  degrees of freedom
## AIC: 3952
## 
## Number of Fisher Scoring iterations: 6
\end{verbatim}

\begin{Shaded}
\begin{Highlighting}[]
\DocumentationTok{\#\#\#\#\#\#\#\#\#\# Negative binomial regression model \#\#\#\#\#\#\#\#\#\#}
\FunctionTok{library}\NormalTok{(MASS)}

\NormalTok{fit1 }\OtherTok{=} \FunctionTok{glm.nb}\NormalTok{(TICKS }\SpecialCharTok{\textasciitilde{}}\NormalTok{ HEIGHT}\SpecialCharTok{*}\NormalTok{YEAR, }\AttributeTok{data =}\NormalTok{ grouseticks, }\AttributeTok{link=}\NormalTok{log)}
\FunctionTok{summary}\NormalTok{(fit1)}
\end{Highlighting}
\end{Shaded}

\begin{verbatim}
## 
## Call:
## glm.nb(formula = TICKS ~ HEIGHT * YEAR, data = grouseticks, link = log, 
##     init.theta = 0.9000852793)
## 
## Deviance Residuals: 
##     Min       1Q   Median       3Q      Max  
## -2.3765  -1.0281  -0.5052   0.2408   3.2440  
## 
## Coefficients:
##                 Estimate Std. Error z value Pr(>|z|)    
## (Intercept)    20.030124   1.827525  10.960  < 2e-16 ***
## HEIGHT         -0.041308   0.004033 -10.242  < 2e-16 ***
## YEAR96        -10.820259   2.188634  -4.944 7.66e-07 ***
## YEAR97        -10.599427   2.527652  -4.193 2.75e-05 ***
## HEIGHT:YEAR96   0.026132   0.004824   5.418 6.04e-08 ***
## HEIGHT:YEAR97   0.020861   0.005571   3.745 0.000181 ***
## ---
## Signif. codes:  0 '***' 0.001 '**' 0.01 '*' 0.05 '.' 0.1 ' ' 1
## 
## (Dispersion parameter for Negative Binomial(0.9001) family taken to be 1)
## 
##     Null deviance: 840.71  on 402  degrees of freedom
## Residual deviance: 418.82  on 397  degrees of freedom
## AIC: 1912.6
## 
## Number of Fisher Scoring iterations: 1
## 
## 
##               Theta:  0.9001 
##           Std. Err.:  0.0867 
## 
##  2 x log-likelihood:  -1898.5880
\end{verbatim}

\begin{Shaded}
\begin{Highlighting}[]
\DocumentationTok{\#\#\#\#\#\#\#\#\#\# Proportional hazard model \#\#\#\#\#\#\#\#\#\#}
\FunctionTok{library}\NormalTok{(survival)}

\CommentTok{\# For data}
\CommentTok{\# install.packages(\textquotesingle{}carData\textquotesingle{})}
\FunctionTok{library}\NormalTok{(carData)}

\NormalTok{?Rossi}

\FunctionTok{Surv}\NormalTok{(Rossi}\SpecialCharTok{$}\NormalTok{week, Rossi}\SpecialCharTok{$}\NormalTok{arrest)}
\end{Highlighting}
\end{Shaded}

\begin{verbatim}
##   [1] 20  17  25  52+ 52+ 52+ 23  52+ 52+ 52+ 52+ 52+ 37  52+ 25  46  28  52+
##  [19] 52+ 52+ 52+ 52+ 24  52+ 52+ 52+ 52+ 52+ 52+ 52+ 52+ 52+ 52+ 52+ 52+ 50 
##  [37] 52+ 52+ 52+ 52+ 52+ 52+ 10  52+ 52+ 52+ 52+ 20  52+ 52+ 52+ 52+ 52+ 50 
##  [55] 52+ 52+ 52+ 52+ 52+ 52+  6  52+ 52+ 52+ 52  52+ 52+ 52+ 49  52+ 52+ 52+
##  [73] 52+ 52+ 52+ 52+ 43  52+ 52+  5  27  52+ 52+ 52+ 22  52+ 52+ 18  52+ 52+
##  [91] 52+ 52+ 52+ 52+ 52+ 24  52+ 52+ 52+ 52+  2  26  52+ 49  52+ 21  48  52+
## [109] 52+ 52+ 52+ 52+ 52+ 52+ 52+ 52+ 52+ 52+  8  52+ 52+ 49  52+ 52+ 52+ 52+
## [127] 52+ 52+ 52+ 52+  8  52+ 52+ 13  52+ 52+ 52+ 52+  8  52+ 33  52+ 52+ 11 
## [145] 52+ 52+ 52+ 52+ 52+ 37  52+ 52+ 52+ 44  52+ 52  52+ 52+ 52+ 52+ 52+ 52+
## [163] 52+ 52+ 52+ 52+ 52+ 52+ 52+ 52+ 52+ 52+  9  17  52+ 52+ 52+ 52+ 52+ 52+
## [181] 16  52+ 52+  3  52+ 52+ 52+ 52+ 52+ 52+ 52+ 52+ 52+ 52+ 45  52+ 52+ 52+
## [199] 52+ 52+ 52+ 28  52+ 16  15  52+ 52+ 52+ 52+ 14  52+ 52+ 52+ 52+ 52+ 52+
## [217] 52+ 52+ 52+ 52+ 52+ 52+ 52+ 52+ 52+  7  52+ 52+ 43  46  40  52+ 14  52+
## [235] 52+  8  52+ 52+ 52+ 52+ 52+ 25  52+ 52+ 17  37  52+ 52+ 52+ 32  52+ 52+
## [253] 52+ 52+ 52+ 52+ 52+ 52+ 12  52+ 18  52+ 52+ 14  52+ 52+ 52+ 52+ 38  52+
## [271] 24  20  32  52+ 52+ 52+ 52+ 52+ 52+ 52+ 52+ 52+ 52+ 52+ 31  20  40  52+
## [289] 52+ 52+ 52+ 42  52+ 26  52+ 52+ 52+ 52+ 52+ 47  52+ 52+ 40  52+ 52+ 52+
## [307] 52+ 21  52+ 52+ 52+ 52+ 52+  1  43  24  11  52+ 52+ 52+ 52+ 33  52+ 46 
## [325] 36  52+ 52+ 18  52+ 52+ 50  52+ 34  52+ 35  52+ 52+ 39   9  52+ 52+ 52+
## [343] 34  52+ 52+ 52+ 44  52+ 52+ 35  30  39  52+ 52+ 52+ 52+ 19  52+ 43  52+
## [361] 48  37  20  52+ 52+ 36  52+ 52+ 52+ 52+ 52+ 52+ 52+ 30  52+ 52+ 52+ 52+
## [379] 52+ 52+ 52+ 52+ 42  52+ 52+ 52+ 52+ 26  40  52+ 52+ 52+ 35  52+ 46  52+
## [397] 49  52+ 52+ 49  52+ 52+ 52+ 52+ 52+ 35  52+ 52+ 52+ 52+ 27  52+ 52+ 52+
## [415] 52  45   4  52  36  52+ 52+  8  15  52+ 19  52+ 12  52+ 52+ 52+ 52+ 52+
\end{verbatim}

\begin{Shaded}
\begin{Highlighting}[]
\NormalTok{fit }\OtherTok{=} \FunctionTok{coxph}\NormalTok{(}\FunctionTok{Surv}\NormalTok{(week,arrest) }\SpecialCharTok{\textasciitilde{}}\NormalTok{ fin }\SpecialCharTok{+}\NormalTok{ age}\SpecialCharTok{+}\NormalTok{ race }\SpecialCharTok{+}\NormalTok{ wexp }\SpecialCharTok{+}\NormalTok{ mar }\SpecialCharTok{+}\NormalTok{ prio, }
            \AttributeTok{data=}\NormalTok{Rossi)}
\FunctionTok{summary}\NormalTok{(fit)}
\end{Highlighting}
\end{Shaded}

\begin{verbatim}
## Call:
## coxph(formula = Surv(week, arrest) ~ fin + age + race + wexp + 
##     mar + prio, data = Rossi)
## 
##   n= 432, number of events= 114 
## 
##                    coef exp(coef) se(coef)      z Pr(>|z|)    
## finyes         -0.37352   0.68831  0.19082 -1.957 0.050295 .  
## age            -0.05640   0.94516  0.02184 -2.583 0.009796 ** 
## raceother      -0.30983   0.73357  0.30780 -1.007 0.314133    
## wexpyes        -0.15331   0.85786  0.21218 -0.723 0.469957    
## marnot married  0.44339   1.55799  0.38136  1.163 0.244958    
## prio            0.09336   1.09785  0.02832  3.296 0.000981 ***
## ---
## Signif. codes:  0 '***' 0.001 '**' 0.01 '*' 0.05 '.' 0.1 ' ' 1
## 
##                exp(coef) exp(-coef) lower .95 upper .95
## finyes            0.6883     1.4528    0.4735    1.0005
## age               0.9452     1.0580    0.9056    0.9865
## raceother         0.7336     1.3632    0.4013    1.3410
## wexpyes           0.8579     1.1657    0.5660    1.3003
## marnot married    1.5580     0.6419    0.7378    3.2898
## prio              1.0979     0.9109    1.0386    1.1605
## 
## Concordance= 0.642  (se = 0.027 )
## Likelihood ratio test= 33.08  on 6 df,   p=1e-05
## Wald test            = 32.01  on 6 df,   p=2e-05
## Score (logrank) test = 33.43  on 6 df,   p=9e-06
\end{verbatim}

\begin{Shaded}
\begin{Highlighting}[]
\CommentTok{\# Estimated survival function}
\FunctionTok{plot}\NormalTok{(}\FunctionTok{survfit}\NormalTok{(fit),}\AttributeTok{ylim=}\FunctionTok{c}\NormalTok{(}\FloatTok{0.6}\NormalTok{,}\DecValTok{1}\NormalTok{),}\AttributeTok{xlab=}\StringTok{"Weeks"}\NormalTok{, }\AttributeTok{ylab=}\StringTok{"Prop.of Not Rearrested"}\NormalTok{)}


\CommentTok{\# Estimated survival functions for financial aid}
\NormalTok{Rossi.fin }\OtherTok{=} \FunctionTok{with}\NormalTok{(Rossi, }\FunctionTok{data.frame}\NormalTok{(}\AttributeTok{fin=}\FunctionTok{c}\NormalTok{(}\DecValTok{0}\NormalTok{, }\DecValTok{1}\NormalTok{), }\AttributeTok{age=}\FunctionTok{rep}\NormalTok{(}\FunctionTok{mean}\NormalTok{(age), }\DecValTok{2}\NormalTok{), }
                                   \AttributeTok{race=}\FunctionTok{rep}\NormalTok{(}\FunctionTok{mean}\NormalTok{(race}\SpecialCharTok{==}\StringTok{\textquotesingle{}other\textquotesingle{}}\NormalTok{),}\DecValTok{2}\NormalTok{), }
                                   \AttributeTok{wexp=}\FunctionTok{rep}\NormalTok{(}\FunctionTok{mean}\NormalTok{(wexp}\SpecialCharTok{==}\StringTok{"yes"}\NormalTok{),}\DecValTok{2}\NormalTok{), }
                                   \AttributeTok{mar=}\FunctionTok{rep}\NormalTok{(}\FunctionTok{mean}\NormalTok{(mar}\SpecialCharTok{==}\StringTok{"not married"}\NormalTok{),}\DecValTok{2}\NormalTok{),}
                                   \AttributeTok{prio=}\FunctionTok{rep}\NormalTok{(}\FunctionTok{mean}\NormalTok{(prio),}\DecValTok{2}\NormalTok{)))}

\FunctionTok{plot}\NormalTok{(}\FunctionTok{survfit}\NormalTok{(fit,}\AttributeTok{newdata=}\NormalTok{Rossi.fin), }\AttributeTok{conf.int=}\ConstantTok{TRUE}\NormalTok{,}
     \AttributeTok{lty=}\FunctionTok{c}\NormalTok{(}\DecValTok{1}\NormalTok{, }\DecValTok{2}\NormalTok{), }\AttributeTok{ylim=}\FunctionTok{c}\NormalTok{(}\FloatTok{0.6}\NormalTok{, }\DecValTok{1}\NormalTok{), }\AttributeTok{col=}\FunctionTok{c}\NormalTok{(}\StringTok{\textquotesingle{}red\textquotesingle{}}\NormalTok{,}\StringTok{\textquotesingle{}blue\textquotesingle{}}\NormalTok{), }
     \AttributeTok{xlab=}\StringTok{"Weeks"}\NormalTok{, }\AttributeTok{ylab=}\StringTok{"Prop. of Not Rearrested"}\NormalTok{)}
\end{Highlighting}
\end{Shaded}

\begin{verbatim}
## Warning in model.frame.default(data = structure(list(fin = c(0, 1), age =
## c(24.5972222222222, : variable 'fin' is not a factor
\end{verbatim}

\begin{verbatim}
## Warning in model.frame.default(data = structure(list(fin = c(0, 1), age =
## c(24.5972222222222, : variable 'race' is not a factor
\end{verbatim}

\begin{verbatim}
## Warning in model.frame.default(data = structure(list(fin = c(0, 1), age =
## c(24.5972222222222, : variable 'wexp' is not a factor
\end{verbatim}

\begin{verbatim}
## Warning in model.frame.default(data = structure(list(fin = c(0, 1), age =
## c(24.5972222222222, : variable 'mar' is not a factor
\end{verbatim}

\begin{Shaded}
\begin{Highlighting}[]
\FunctionTok{legend}\NormalTok{(}\StringTok{"bottomleft"}\NormalTok{, }\AttributeTok{legend=}\FunctionTok{c}\NormalTok{(}\StringTok{"fin = no"}\NormalTok{,}\StringTok{"fin = yes"}\NormalTok{), }
       \AttributeTok{lty=}\FunctionTok{c}\NormalTok{(}\DecValTok{1}\NormalTok{ ,}\DecValTok{2}\NormalTok{),}\AttributeTok{col=}\FunctionTok{c}\NormalTok{(}\StringTok{\textquotesingle{}red\textquotesingle{}}\NormalTok{,}\StringTok{\textquotesingle{}blue\textquotesingle{}}\NormalTok{))}
\end{Highlighting}
\end{Shaded}

\includegraphics{SM_Example1_files/figure-latex/unnamed-chunk-1-9.pdf}

\end{document}
